% This is samplepaper.tex, a sample chapter demonstrating the
% LLNCS macro package for Springer Computer Science proceedings;
% Version 2.21 of 2022/01/12
%
\documentclass[runningheads]{llncs}
%
\usepackage[T1]{fontenc}
% T1 fonts will be used to generate the final print and online PDFs,
% so please use T1 fonts in your manuscript whenever possible.
% Other font encondings may result in incorrect characters.
%
\usepackage{hyperref}
\usepackage{graphicx}
% \usepackage{quiver}
\usepackage{proof}
% \usepackage{amsmath,amssymb,amsfonts}
\usepackage{amsmath,amsfonts,MnSymbol}
\usepackage{}
\usepackage{tikz-cd}
\usepackage{float}
% \tikzcdset{scale cd/.style={every label/.append style={scale=#1},
%     cells={nodes={scale=#1}}}}
\tikzcdset{row sep/normal=0.22cm}
\tikzcdset{column sep/normal=0.02cm}
% Used for displaying a sample figure. If possible, figure files should
% be included in EPS format.
%
% If you use the hyperref package, please uncomment the following two lines
% to display URLs in blue roman font according to Springer's eBook style:
\usepackage{color}
\renewcommand\UrlFont{\color{blue}\rmfamily}
\urlstyle{rm}
%
% \usepackage{amsmath,amssymb,amsfonts}%

%% macros for typesetting
\newcommand{\udl}[1]{\underline{#1}}
\newcommand{\proofbox}[1]{\begin{array}{c} #1 \end{array}}

%% macros for math symbols
\newcommand{\ot}{\otimes}
\newcommand{\cdast}{\circledast}
\newcommand{\Larr}{\Leftarrow}
\newcommand{\Ls}{\Leftarrow^*}
\newcommand{\Rarr}{\Rightarrow}
\newcommand{\Rs}{\Rightarrow^*}
\newcommand{\btleft}{\blacktriangleleft}
\newcommand{\btright}{\blacktriangleright}
\newcommand{\sls}{\slash}
\newcommand{\bsls}{\backslash}
\newcommand{\mc}[1]{\mathcal{#1}}
\newcommand{\mf}[1]{\mathsf{#1}}
\newcommand{\vars}[1]{\mf{var} (#1)}
\newcommand{\gs}[1]{\sigma_{X} (#1)}
\newcommand{\GG}{\Gamma}
\newcommand{\Gg}{\gamma}
\newcommand{\GD}{\Delta}
\newcommand{\Gd}{\delta}
\newcommand{\GL}{\Lambda}
\newcommand{\Gl}{\lambda}
\newcommand{\GO}{\Omega}
\newcommand{\I}{\mf{I}}
\newcommand{\rcons}{\nersquigarrow::}
\newcommand{\lcons}{\nwrsquigarrow::}

%% macros for acronyms 
\newcommand{\MIP}{\textsf{MIP}}
\newcommand{\MIPeq}{{\textsf{MIP}{\circeq}}}
\newcommand{\FL}{$\mathtt{FL}$}
\renewcommand{\L}{$\mathtt{L}$}
\newcommand{\LPfree}{$\mathtt{L}_{(\Rarr, \Larr)}$}
\newcommand{\NL}{$\mathtt{NL}$}

%% macros for derivations
\newcommand{\vd}{\vdash}
\newcommand{\Vd}{\Vdash}
\newcommand{\ax}{\mathsf{ax}}
\newcommand{\tl}{{\otimes}\mathsf{L}}
\newcommand{\tr}{{\otimes}\mathsf{R}}
\newcommand{\Ll}{{\Larr}\mathsf{L}}
\newcommand{\Lsl}{{\Ls}\mathsf{L}}
\newcommand{\Lr}{{\Larr}\mathsf{R}}
\newcommand{\Lsr}{{\Ls}\mathsf{R}}
\newcommand{\Rl}{{\Rarr}\mathsf{L}}
\newcommand{\Rsl}{{\Rs}\mathsf{L}}
\newcommand{\Rr}{{\Rarr}\mathsf{R}}
\newcommand{\Rsr}{{\Rs}\mathsf{R}}
\newcommand{\cut}{\mf{cut}}
\newcommand{\cuts}{\mf{cut}^*}

%% commands for Agda stuff
\newcommand{\At}{\mathsf{At}}
\newcommand{\at}{\mathsf{at}}
\newcommand{\Fma}{\mathsf{Fma}}
\newcommand{\data}{\mathsf{data}}
\newcommand{\Tree}{\mathsf{Tree}}
\newcommand{\Path}{\mathsf{Path}}
\newcommand{\pathT}[1]{\mathsf{Path} ~ #1}
\newcommand{\append}{+\!\!+}
\newcommand{\Sub}{\mathsf{sub}}
\newcommand{\sub}[2]{\mathsf{sub} ~ #1 ~ #2}
\newcommand{\where}{\mathsf{where}}
\newcommand{\Set}{\mathsf{Set}}
\newcommand{\record}{\mathsf{record}}
\newcommand{\field}{\mathsf{field}}
\newcommand{\subst}{\mathsf{subst}}
\newcommand{\Same}{\mathsf{Same}}
\newcommand{\ContainsLeft}{{\in}\mathsf{Left}}
\newcommand{\ContainsRight}{{\in}\mathsf{Right}}
\newcommand{\LeftRight}{\mathsf{Disj}}
\newcommand{\subcases}{\mathsf{SubEq}}
\newcommand{\oneeqtwo}{\mathsf{case}_1}%{\mathsf{1{\equiv}2}}
\newcommand{\twogtLone}{\mathsf{case}_2}%{\mathsf{2{>}L1}}
\newcommand{\twogtRone}{\mathsf{case}_3}%{\mathsf{2{>}R1}}
\newcommand{\onegtLtwo}{\mathsf{case}_4}%{\mathsf{1{>}L2}}
\newcommand{\onegtRtwo}{\mathsf{case}_5}%{\mathsf{1{>}R2}}
\newcommand{\oneLtwoR}{\mathsf{case}_6}%{\mathsf{1\sls \bsls 2}}
\newcommand{\oneRtwoL}{\mathsf{case}_7}%{\mathsf{2\sls \bsls 1}}
\newcommand{\subeq}{\mathsf{subeq}}
\newcommand{\inT}{\in^{\mf{T}}}
\newcommand{\conc}{{~ ++ ~}}

%% macros for names
\newcommand{\cub}{\v{C}ubri{\'c}}

%% macros for text colors
\newcommand{\tred}[1]{\textcolor{red}{#1}}

\newcommand{\niccolo}[1]{\textcolor{red}{NV: #1}}
\newcommand{\cheng}[1]{\textcolor{blue}{CSW: #1}}

\begin{document}
%
\title{Proof-relevant Interpolation of Lambek Calculi}
%
\titlerunning{Proof-relevant Interpolation of Lambek Calculi}
% If the paper title is too long for the running head, you can set
% an abbreviated paper title here
%
\author{Niccol{\`o} Veltri\orcidID{0000-0002-7230-3436} \and
Cheng-Syuan Wan \orcidID{0000-0003-2053-1688}}
%
\authorrunning{N. Veltri and C.-S. Wan}
% First names are abbreviated in the running head.
% If there are more than two authors, 'et al.' is used.
%
\institute{Tallinn University of Technology, Tallinn, Estonia
\\
\email{\{niccolo,cswan\}@cs.ioc.ee}}
%
\maketitle              % typeset the header of the contribution
%
\begin{abstract}
  This paper studies proof-relevant interpolation for Lambek calculus and its product-free variant.
\keywords{Lambek calculus \and Agda \and cut admissibility \and Maehara interpolation \and proof-relevant interpolation}
\end{abstract}
%
%
%

\section{Introduction}
From the perspective of sequent calculus, substructural logics are characterized by the absence of at least one structural rule. A notable instance is Lambek's syntactic calculus \cite{lambek:mathematics:58}, which forbids the structural rules of weakening, contraction and exchange. Its non-associative variant, also introduced by Lambek \cite{Lambek1961}, further disallows associativity.
% also introduced by Lambek \cite{Lambek1961}, disallows associativity as well.
The Lambek calculus (both its associative and nonassociative variants) has been extensively studied in the literature, especially for its linguistic applications \cite{moot:categorial:2012}.
%While Lambek calculus has different presentations, including axiomatic (or syntactic), natural deduction, and sequent calculus, we focus primarily on the proof-theoretic aspect, specifically the sequent calculus formulation.

In its sequent calculus presentation, sequents in the associative Lambek calculus (\L) are of the form $\GG \vd C$, where $\GG$ is a \emph{list} of formulae and $C$ is a single formula.


\begin{enumerate}
  \item calculi
  \item cut admissibility, just theorems
  \item Maehara interpolation, kind of also well known but pay some more attention on multi-interpolation
  \item interpolation will then induce a calculus decorated with interpolant triples
  \begin{enumerate}
    \item show some example rules
  \end{enumerate}
  \item equivalence of derivations
  \item proof-relevant interpolation for both, i.e. cut is the left inverse of interpolation
  \item well-definedness of interpolation procedure wrt. equivalence of derivations, in the nonassociative case, it was kind of trivial but in the associative case, there are something interesting, 
  \begin{enumerate}
    \item first, not all interpolant formulae from a pair of equivalent derivations are the same, they can be logically equivalent or even weaker, only one way of derivability.
    \item then a good notion to characterize them will be the zig-zag equivalence.
    \item After that, we show the well-definedness for both assoc Lam and prod-free Lam with the examples appear in both calculi, i.e.\ the curry and uncurry formulae, $(E \ot D) \Rarr F$ and $D \Rarr (E \Rarr F)$ for  ->L->L-assoc, and $(F \Larr E) \ot F$ and $F \Larr (D \Rarr E)$ for ->L<-L-comm.
  \end{enumerate}  
  \item maybe it will be good to think about computational interpolation of this behavior but I am not sure about that.
\end{enumerate}

\section{Sequent Calculus of Lambek calculus}\label{sec:seqcalc}
Formulae of \L\ are inductively generated by the grammar $A, B ::= X \ | \ A \Larr B \ | \ B \Rarr A \ | \ A \ot B$, where $X$ is drawn from a set $\mathsf{At}$ of atomic formulae,

Derivations of \L\ are inductively generated by the following rules:

\begin{equation}\label{eq:seqcalc}
\begin{array}{c}
  \infer[\ax]{A \vd A}{}
  \quad
  \infer[\Rr]{\GG \vd A \Rarr B}{A , \GG \vd B} 
  \quad
  \infer[\Lr]{\GG \vd B \Larr A}{\GG , A \vd B}
  \\[7pt]
  \infer[\Rl]{\GG , A \Rarr B , \GL \vd C}{
    \GD \vd A
    &
    \GG , B , \GL \vd C
  }
  \quad
  \infer[\Ll]{\GG , B \Larr A , \GL \vd C}{
    U \vd A
    &
    \GG , B , \GL \vd C
  }
  \\[7pt]
  \infer[\tr]{\GG , \GD \vd A \ot B}{
    \GG \vd A
    &
    \GD \vd B
  }
  \quad
  \infer[\tl]{\GG , A \ot B , \GD \vd C}{
    \GG , A , B , \GD \vd C
  }
\end{array}
\end{equation}

It is well-known that both cut admissibility \cite{moot:categorial:2012} and Maehara interpolation \cite{roorda1991,ono:proof:nonclassical:1998,moot:categorial:2012} hold in \L. Here we state the properties and present the some cases in the proof that will be used in later sections.

\begin{theorem}[Cut admissibility]\label{thm:cut}
  The cut rule
  \[
  \infer[\cut]{\GG , \GD , \GL \vd C}{
    \deduce{\GD \vd A}{f}
    &
    \deduce{\GG , A , \GL \vd C}{g}
  }
  \]
   is admissible in \L.
\end{theorem}
\begin{proof}
  Proof proceeds by structural induction on the second premise and the complexity of the cut formula.
  % The cases are straightforward when the second premise is $\ax$, a conclusion of a right rule, or a conclusion of a left rule but the cut formula is different from the principal formula.
  If the second premise $\ax$, then $\cut\ (f , \ax) = f$. 
  If the second premise is a conclusion of a right rule, or a conclusion of a left rule but the cut formula is different from the principal formula, then we permute the corresponding rule down and continue recursively.
  
  We perform a further induction on the first premise when the cut formula coincides with the principal formula of a left rule.
  If the principal formula is not the cut formula in the first premise, then we permute the rule down and continue recursively.
  If the principal formula is the cut formula, then we do the following:
  \[
    \begin{array}{l}
      \proofbox{
        \infer[\cut]{\GG , \GD_0 , \GD_1 , \GL \vd C}{
        \infer[\tr]{\GD_0 , \GD_1 \vd A' \ot B'}{
          \deduce{\GD_0 \vd A'}{f'}
          &
          \deduce{\GD_1 \vd B'}{f''}
        }
        &
        \infer[\tl]{\GG , A' \ot B' \GL \vd C}{
          \deduce{\GG , A' , B' \GL \vd C}{g'}
        }
      }
      }
      \\
      \hspace*{4cm}=\qquad
      \proofbox{
        \infer[\cut]{\GG , \GD_0 , \GD_1 , \GL \vd C}{
        \deduce{\GD_0 \vd A'}{f'}
        &
        \infer[\cut]{\GG , A' , \GD_1 , \GL \vd C}{
          \deduce{\GD_1 \vd B'}{f''}
          &
          \deduce{\GG , A' , B' \GL \vd C}{g'}
        }
      }
      }
      \\[5pt]
      \proofbox{
        \infer[\cut]{\GG_0 , \GG_1 , \GD , \GL \vd C}{
        \infer[\Rr]{\GD \vd A' \Rarr B'}{
          \deduce{A' , \GD \vd B'}{f'}
        }
        &
        \infer[\Rl]{\GG_0 , \GG_1 , A' \Rarr B' , \GL \vd C}{
          \deduce{\GG_1 \vd A'}{g'}
          &
          \deduce{\GG_0 , B , \GL \vd C}{g''}
        }
      }
      }
      \\
      \hspace*{4cm}=\qquad
      \proofbox{
        \infer[\cut]{\GG_0 , \GG_1 , \GD , \GL \vd C}{
          \deduce{\GG_1 \vd A'}{g'}
          &
          \infer[\cut]{\GG_0 , A' , \GD , \GL \vd C}{
            \deduce{A' , \GD \vd B'}{f'}
            &
            \deduce{\GG_0 , B , \GL \vd C}{g''}
          }
        }
      }
    \end{array}
  \]
  The case for $\cut \ (\Lr \ f' , \Ll \ (g' ,g''))$ is similar.
\end{proof}

\begin{theorem}[Maehara interpolation]\label{thm:MIP}
  Given a derivation $f : \GG , \GD , \GL \vd C$, there exist a formula $D$ and two derivations $g : \GG, D, \GL \vdash C$ and $h : \GD \vdash D$ such that $\vars{D} \subseteq \vars{\GD} \cap \vars{\GG, \GL, C}$
\end{theorem}
\begin{proof}
  Proof proceeds by structural induction on the derivation $f$.
  We show the most complicated case, \ $f = \Rl \ (f' , f'')$, which will be recalled multiple times in the rest of the paper:
  \[
    \infer[\Rl]{\GG' , \GD' , A \Rarr B , \GL' \vd C}{
        \deduce{\GD' \vd A}{f'}
        &
        \deduce{\GG' , B, \GL' \vd C}{f''}
    }
  \]
  There are six subcases. The first four cases are that $\GD$ is contained in either $\GG'$, $\GD'$ or $\GL'$ or $\GD$ contains $\GD' , A \Rarr B$. In these cases, we apply the inductive hypothesis appropriately to obtain desired interpolant formula and derivations.
  For example, if $\GD$ is contained in $\GD'$, i.e.\ $\GD' = \GD'_0 , \GD , \GD'_1$, then by inductive hypothesis on $f'$, there exist a formula $D$ and two derivations $g' : \GD'_0 , D , \GD'_1 \vd A$ and $h' : \GD \vd D$. In this case the desired interpolant formula is $D$ and two desired derivations are constructed as follows: 
  \[
    \begin{array}{c}
    \proofbox{
      \infer[\Rl]{\GG' , \GD'_0 , D , \GD'_1 , A \Rarr B , \GL' \vd C}{
        \deduce{\GD'_0 , D , \GD'_1 \vd A}{g'}
        &
        \deduce{\GG' , B, \GL' \vd C}{f''}
      }
    }
      \qquad
      \proofbox{
        \deduce{\GD \vd D}{h'}
      }
    \end{array}
  \]
  The variable condition holds automatically.

  We show the rest two interesting cases. The first one is when $\GG' = \GG'_0 , \GG'_1$, $\GD' = \GD'_0 , \GD'_1$ and $\GD = \GG'_1 , \GD'_0$ ($\GD$ splits both $\GG'$ and $\GD'$). In this case, we apply inductive hypothesis on $f'$ and $f''$ respectively to obtain
      \begin{itemize}
      \item a formula $D$,
      \item derivations $g' : D , \GD'_1 \vd A$ and $h' : \GD'_0 \vd D$ such that
      \item $\vars{D} \subseteq \vars{\GD'_0} \cap \vars{\GD'_1 , A}$ and
    \end{itemize}
      \begin{itemize}
      \item a formula $E$,
      \item derivations $g'' : \GG'_0 , E , B , \GL' \vd C$ and $h'' : \GG'_1 \vd E$ such that
      \item $\vars{E} \subseteq \vars{\GG'_1} \cap \vars{\GG'_0 , B , \GL' , C}$.
    \end{itemize}
  The desired interpolant formula in this case is $E \ot D$ and the desired derivations are constructed as follows:
  \[
  \begin{array}{c}
    \proofbox{
      \infer[\tl]{\GG'_0 , E \ot D  , A \Rarr B , \GL' \vd C}{
        \infer[\Rl]{\GG'_0 , E , D  , A \Rarr B , \GL' \vd C}{
          \deduce{D , \GD'_1 \vd A}{g'}
          &
          \deduce{\GG'_0 , E , B , \GL' \vd C}{g''}
        }
      }
    }
    \qquad
    \proofbox{
      \infer[\tr]{\GG'_1 , \GD'_0 \vd E \ot D}{
        \deduce{\GG'_1 \vd E}{h''}
        &
        \deduce{\GD'_0 \vd D}{h'}
      }
    }
  \end{array}
  \]
  The variable condition is easy to check.

  The second interesting case is when $\GD' = \GD'_0 , \GD'_1$, $\GL' = \GL'_0 , \GD'_1$ and $\GD = \GD'_1 , A \Rarr B , \GL'_0$ ($\GD$ splits both $\GD'$ and $\GL'$ and contains $A \Rarr B$).
  In this case, we apply inductive hypothesis on $f'$ and $f''$ respectively to obtain
      \begin{itemize}
      \item a formula $D$,
      \item derivations $g' : D , \GD'_1 \vd A$ and $h' : \GD'_0 \vd D$ such that
      \item $\vars{D} \subseteq \vars{\GD'_0} \cap \vars{\GD'_1 , A}$ and
    \end{itemize}
      \begin{itemize}
      \item a formula $E$,
      \item derivations $g'' : \GG' , E , \GL'_1 \vd C$ and $h'' : B , \GL'_0 \vd E$ such that
      \item $\vars{E} \subseteq \vars{B , \GL'_0} \cap \vars{\GG' , \GL'_1 , C}$.
    \end{itemize}
    The desired interpolant formula is $D \Rarr E$ and the desired derivations are constructed as follows:
    \[
    \begin{array}{c}
      \proofbox{
        \infer[\Rl]{\GG' , \GD'_0 , D \Rarr E , \GL'_1 \vd C}{
          \deduce{\GD'_0 \vd D}{h'}
          &
          \deduce{\GG' , E , \GL'_1 \vd C}{g''}
        }
      }
      \qquad
      \proofbox{
        \infer[\Rr]{\GD'_1 , A \Rarr B , \GL'_0 \vd D \Rarr E}{
          \infer[\Rl]{D , \GD'_1 , A \Rarr B , \GL'_0 \vd E}{
            \deduce{D , \GD'_1 \vd A}{g'}
            &
            \deduce{B , \GL'_0 \vd E}{h''}
          }
        }
      }
    \end{array}
    \]
    The variable condition is easy to check as well.
\end{proof}
We call $(D , g , h)$ the interpolant triple of $f : \GG , \GD , \GL \vd C$ if it is obtained by applying Maehara interpolation procedure on $f$.

\section{Equivalence of Derivations}\label{sec:equiv:derivation}
In sequent calculus, it is often the case that a sequent has different derivations. 
However, some of these derivations are ``morally'' the same and they differ syntactically because of the bureaucracies of the rules.
For example, the sequent $\GG , \GD , A \Rarr B , \GL \vd A' \Rarr B'$ in \L\ has two different proofs from derivations $f : \GD \vd A$ and $f' : A' , \GG , B , \GL \vd B'$:
\[
\begin{array}{c}
  \proofbox{\infer[\Rl]{\GG , \GD , A \Rarr B , \GL \vd A' \Rarr B'}{
    \deduce{\GD \vd A}{f}
    &
    \infer[\Rr]{\GG , B , \GL \vd A' \Rarr B'}{
      \deduce{A' , \GG , B , \GL \vd B'}{f'}
    }
  }
  }
  \quad
  \proofbox{\infer[\Rr]{\GG , \GD , A \Rarr B , \GL \vd A' \Rarr B'}{
    \infer[\Rl]{A' , \GG , \GD , A \Rarr , \GL \vd B'}{
      \deduce{\GD \vd A}{f}
      &
      \deduce{A' , \GG , B , \GL \vd B'}{f'}
    }
  }
  }
\end{array}
\]
Intuitively, these two derivations should be considered equal in \L\ since the order of rule applications does not provide new information.

The set of derivations in \L\ are then quotiented by the equivalence relation $\circeq$ that is defined by a collection of equations of derivations.
The collection contains the following data: ($i$) the relation is an equivalence relation; ($ii$) the congruence of each logical rule, e.g.\ $\Rr \ f \circeq \Rr \ f'$ if $f \circeq f'$ and $\Rl ~ (f , f_1) \circeq \Rl ~ (f' , f'_1)$ if $f \circeq f'$ and $f_1 ~ f'_1$; ($iii$) $\eta$-expansion of $\ax$, i.e.\ an instance of $\ax$ of a compound formula is equivalent to the expansion with left and right rules applying on instances of $\ax$ of simpler formulae, e.g.\ $\ax$ of $A \Rarr B$ expands to $\Rr \ (\Rl \ (\ax ,\ax))$ with $\ax$ of $A$ and $\ax$ of $B$ in Figure \ref{fig:conv}; ($iv$) left and right introduction rules can always be permuted, e.g.\ $\tr$ permutes with $\Rl$ in Figure \ref{fig:conv}; ($v$) consecutive application of left rules is associative, e.g.\ two $\Rl$ instances permute with each other in Figure \ref{fig:conv}, where $B$ and $B'$ are in \emph{different} sequents; ($vi$) consecutive application of left rules is commutative, e.g.\ two $\Rl$ instances permute with each other in Figure \ref{fig:conv}, where $B$ and $B'$ are in the \emph{same} sequent.
\begin{figure}[t]
\normalsize
  % \[
  % \begin{array}{l}
  %   \begin{array}{rcl}
  %   \proofbox{
  %   \infer[\ax]{A \Rarr B \vd A \Rarr B}{}
  % }
  % &\circeq&
  % \proofbox{
  %   \infer[\Rr]{A \Rarr B \vd A \Rarr B}{
  %   \infer[\Rl]{A , A \Rarr B \vd B}{
  %     \infer{A \vd A}{}
  %     &
  %     \infer{B \vd B}{}
  %   }
  % }
  % }
  % \\[0.8cm]
  % \proofbox{\infer[\tr]{\GG , \GD , A \Rarr B , \GL , \GO \vd A' \ot B'}{
  %   \infer[\Rl]{\GG , \GD A \Rarr B , \GL \vd A'}{
  %     \deduce{\GD \vd A}{f}
  %     &
  %     \deduce{\GG , B , \GD \vd A'}{f'}
  %   }
  %   &
  %   \deduce{\GO \vd B'}{f''}
  % }
  % }
  % &\circeq&
  % \proofbox{\infer[\Rl]{\GG , \GD , A \Rarr B , \GL , \GO \vd A' \ot B'}{
  %   \deduce{\GD \vd A}{f}
  %   &
  %   \infer[\tr]{\GG , B , \GL , \GO \vd A' \ot B'}{
  %     \deduce{\GG , B , \GD \vd A'}{f'}
  %     &
  %     \deduce{\GO \vd B'}{f''}
  %   }
  % }
  % }
  % \end{array}
  % \\[0.8cm]
  % \begin{array}{c}
  %     \proofbox{
  %   \infer[\Rl]{\GG , \GD_0 , A \Rarr B , \GL , \GD_1 , A' \Rarr B', \GO \vd C}{
  %     \deduce{\GD_0 \vd A}{f}
  %     &
  %     \infer[\Rl]{\GG , B , \GL , \GD_1 , A' \Rarr B', \GO \vd C}{
  %       \deduce{\GD_1 \vd A'}{f'}
  %       &
  %       \deduce{\GG , B , \GL , \GD_1 , B', \GO \vd C}{f''}
  %     }
  %   }
  % }
  % \\
  % \hspace*{4.5cm}
  % \circeq
  % \quad
  % \proofbox{
  %   \infer[\Rl]{\GG , \GD_0 , A \Rarr B , \GL , \GD_1 , A' \Rarr B', \GO \vd C}{
  %     \deduce{\GD_1 \vd A'}{f'}
  %     &
  %     \infer[\Rl]{\GG , \GD_0 , B , \GL , B', \GO \vd C}{
  %       \deduce{\GD_0 \vd A}{f}
  %       &
  %       \deduce{\GG , B , \GL , \GD_1 , B', \GO \vd C}{f''}
  %     }
  %   }
  % }
  % \\[0.8cm]
  % \proofbox{
  %   \infer[\Rl]{\GO_0 , \GG , \GD , A \Rarr B , \GL , A' \Rarr B', \GO_1 \vd C}{
  %     \deduce{\GD \vd A}{f}
  %     &
  %     \infer[\Rl]{\GO_0 , \GG , B , \GL , A' \Rarr B' , \GO_1 \vd C}{
  %       \deduce{\GG , B , \GL \vd A'}{f'}
  %       &
  %       \deduce{\GO_0 , B' , \GO_1 \vd C}{f''}
  %     }
  %   }
  % }
  % \\
  % \hspace*{4.5cm}
  % \circeq
  % \quad
  % \proofbox{
  %   \infer[\Rl]{\GO_0 , \GG , \GD , A \Rarr B , \GL , A' \Rarr B', \GO_1 \vd C}{
  %     \infer[\Rl]{\GG , \GD , A \Rarr B , \GL \vd A'}{
  %       \deduce{\GD \vd A}{f}
  %       &
  %       \deduce{\GG , B , \GL \vd A'}{f'}
  %     }
  %     &
  %     \deduce{\GO_0 , B' , \GO_1 \vd C}{f''}
  %   }
  % }
  % \end{array}
  % \end{array}
  % \]
\[  
\begin{array}{l}
  % \multicolumn{1}{c}{
  %   \proofbox{
  %     \infer[\ax]{A \ot B \vd A \ot B}{}
  %   }
  % \circeq
  % \proofbox{
  %   \infer[\tl]{A \ot B \vd A \ot B}{
  %   \infer[\tr]{A , B \vd A \ot B}{
  %     \infer{A \vd A}{}
  %     &
  %     \infer{B \vd B}{}
  %   }
  % }
  % }
  % }
  % \\[0.8cm]
  % \multicolumn{1}{c}{
  %   \proofbox{
  %   \infer[\ax]{A \Rarr B \vd A \Rarr B}{}
  % }
  % \circeq
  % \proofbox{
  %   \infer[\Rr]{A \Rarr B \vd A \Rarr B}{
  %   \infer[\Rl]{A , A \Rarr B \vd B}{
  %     \infer[\ax]{A \vd A}{}
  %     &
  %     \infer[\ax]{B \vd B}{}
  %   }
  % }
  % }
  % % \quad
  % % \proofbox{
  % %   \infer[\ax]{B \Larr A \vd B \Larr A}{}
  % % }
  % % \circeq
  % % \proofbox{
  % %   \infer[\Rr]{B \Larr A \vd B \Larr A}{
  % %   \infer[\Rl]{B \Larr A , A \vd B}{
  % %     \infer{A \vd A}{}
  % %     &
  % %     \infer{B \vd B}{}
  % %   }
  % % }
  % % }
  % }
  % \\[0.8cm]
  % \multicolumn{1}{c}{
  % \proofbox{\infer[\tr]{\GG , \GD , A \Rarr B , \GL , \GO \vd A' \ot B'}{
  %   \infer[\Rl]{\GG , \GD , A \Rarr B , \GL \vd A'}{
  %     \deduce{\GD \vd A}{f}
  %     &
  %     \deduce{\GG , B , \GD \vd A'}{f'}
  %   }
  %   &
  %   \deduce{\GO \vd B'}{f''}
  % }
  % }
  % \circeq
  % \proofbox{\infer[\Rl]{\GG , \GD , A \Rarr B , \GL , \GO \vd A' \ot B'}{
  %   \deduce{\GD \vd A}{f}
  %   &
  %   \infer[\tr]{\GG , B , \GL , \GO \vd A' \ot B'}{
  %     \deduce{\GG , B , \GD \vd A'}{f'}
  %     &
  %     \deduce{\GO \vd B'}{f''}
  %   }
  % }
  % }
  % }
  \multicolumn{1}{c}{
    \begin{array}{rcl}
    \proofbox{
    \infer[\ax]{A \Rarr B \vd A \Rarr B}{}
  }
  &\circeq&
  \proofbox{
    \infer[\Rr]{A \Rarr B \vd A \Rarr B}{
    \infer[\Rl]{A , A \Rarr B \vd B}{
      \infer[\ax]{A \vd A}{}
      &
      \infer[\ax]{B \vd B}{}
    }
  }
  }
  \\[5pt]
  \proofbox{\infer[\tr]{\GG , \GD , A \Rarr B , \GL , \GO \vd A' \ot B'}{
    \infer[\Rl]{\GG , \GD , A \Rarr B , \GL \vd A'}{
      \deduce{\GD \vd A}{f}
      &
      \deduce{\GG , B , \GL \vd A'}{f'}
    }
    &
    \deduce{\GO \vd B'}{f''}
  }
  }
  &\circeq&
  \proofbox{\infer[\Rl]{\GG , \GD , A \Rarr B , \GL , \GO \vd A' \ot B'}{
    \deduce{\GD \vd A}{f}
    &
    \infer[\tr]{\GG , B , \GL , \GO \vd A' \ot B'}{
      \deduce{\GG , B , \GL \vd A'}{f'}
      &
      \deduce{\GO \vd B'}{f''}
    }
  }
  }
    \end{array}
  }
  \\[5pt]
  \proofbox{
    \infer[\Rl]{\GO_0 , \GG , \GD , A \Rarr B , \GL , A' \Rarr B', \GO_1 \vd C}{
      \deduce{\GD \vd A}{f}
      &
      \infer[\Rl]{\GO_0 , \GG , B , \GL , A' \Rarr B' , \GO_1 \vd C}{
        \deduce{\GG , B , \GL \vd A'}{f'}
        &
        \deduce{\GO_0 , B' , \GO_1 \vd C}{f''}
      }
    }
  }
  \\
  \hspace*{4cm}
  \circeq
  \quad
  \proofbox{
    \infer[\Rl]{\GO_0 , \GG , \GD , A \Rarr B , \GL , A' \Rarr B', \GO_1 \vd C}{
      \infer[\Rl]{\GG , \GD , A \Rarr B , \GL \vd A'}{
        \deduce{\GD \vd A}{f}
        &
        \deduce{\GG , B , \GL \vd A'}{f'}
      }
      &
      \deduce{\GO_0 , B' , \GO_1 \vd C}{f''}
    }
  }
  \\[5pt]
   \proofbox{
    \infer[\Rl]{\GG , \GD_0 , A \Rarr B , \GL , \GD_1 , A' \Rarr B', \GO \vd C}{
      \deduce{\GD_0 \vd A}{f}
      &
      \infer[\Rl]{\GG , B , \GL , \GD_1 , A' \Rarr B', \GO \vd C}{
        \deduce{\GD_1 \vd A'}{f'}
        &
        \deduce{\GG , B , \GL , B', \GO \vd C}{f''}
      }
    }
  }
  \\
  \hspace*{4cm}
  \circeq
  \quad
  \proofbox{
    \infer[\Rl]{\GG , \GD_0 , A \Rarr B , \GL , \GD_1 , A' \Rarr B', \GO \vd C}{
      \deduce{\GD_1 \vd A'}{f'}
      &
      \infer[\Rl]{\GG , \GD_0 , A \Rarr B , \GL , B', \GO \vd C}{
        \deduce{\GD_0 \vd A}{f}
        &
        \deduce{\GG , B , \GL , B', \GO \vd C}{f''}
      }
    }
  }
\end{array}
\]
\caption{Examples of $\eta$-expansion and permutative conversions}
\label{fig:conv}
\end{figure}

\begin{lemma}\label{lem:left:rules:cut}
  The following equations of derivations are admissible in \L:
  % \[
  % \begin{array}{l}
  %   \multicolumn{1}{c}{
  %     \proofbox{
  %     \infer[\cut]{\GG , A , \GL \vd C}{
  %       \infer[\ax]{A \vd A}{}
  %       &
  %       \deduce{\GG , A , \GL \vd C}{g}
  %     }
  %   }
  %   \quad\circeq\quad
  %   \proofbox{
  %     \deduce{\GG , A , \GL \vd C}{g}
  %   }
  %   }
  %   \\[5pt]
  %   \proofbox{
  %     \infer[\cut]{\GG , \GD_0 , A \ot B , \GD_1 , \GL \vd C}{
  %       \infer[\tl]{\GD_0 , A \ot B , \GD_1 \vd A'}{
  %         \deduce{\GD_0 , A , B , \GD_1 \vd A'}{f}
  %       }
  %       &
  %       \deduce{\GG , A' , \GL \vd C}{g}
  %     }
  %   }
  %   \\
  %   \hspace*{4cm}
  %   \circeq\quad
  %   \proofbox{
  %     \infer[\tl]{\GG , \GD_0 , A \ot B , \GD_1 , \GL \vd C}{
  %     \infer[\cut]{\GG , \GD_0 , A , B , \GD_1 , \GL \vd C}{
  %       \deduce{\GD_0 , A , B , \GD_1 \vd A'}{f}
  %       &
  %       \deduce{\GG , A' , \GL \vd C}{g}
  %     }
  %   }
  %   }
  %   \\[5pt]
  %   \proofbox{
  %     \infer[\cut]{\GG_0 , \GG_1 , \GD , A \Rarr B , \GL_0 , \GL_1 \vd C}{
  %       \infer[\Rl]{\GG_1 , \GD , A \Rarr B , \GL_0 \vd A'}{
  %         \deduce{\GD \vd A}{f'}
  %         &
  %         \deduce{\GG_1 , B , \GL_0 \vd C}{f''}
  %       }
  %       &
  %       \deduce{\GG_0 , A', \GL_1 \vd C}{g}
  %     }
  %   }
  %   \\
  %   \hspace*{4cm}
  %   \circeq\quad
  %   \proofbox{
  %     \infer[\Rl]{\GG_0 , \GG_1 , \GD , A \Rarr B , \GL_0 , \GL_1 \vd C}{
  %       \deduce{\GD \vd A}{f'}
  %       &
  %       \infer[\cut]{\GG_0 , \GG_1 , B , \GL_0 , \GL_1 \vd C}{
  %         \deduce{\GG_1 , B , \GL_0 \vd C}{f''}
  %         &
  %         \deduce{\GG_0 , A', \GL_1 \vd C}{g}
  %       }
  %     }
  %   }
  %   \\[5pt]
  %   \proofbox{
  %     \infer[\cut]{\GG_0 , \GG_1 , B \Larr A , \GD , \GL_0 , \GL_1 \vd C}{
  %       \infer[\Ll]{\GG_1 , B \Larr A , \GD , \GL_0 \vd A'}{
  %         \deduce{\GD \vd A}{f'}
  %         &
  %         \deduce{\GG_1 , B , \GL_0 \vd C}{f''}
  %       }
  %       &
  %       \deduce{\GG_0 , A', \GL_1 \vd C}{g}
  %     }
  %   }
  %   \\
  %   \hspace*{4cm}
  %   \circeq\quad
  %   \proofbox{
  %     \infer[\Ll]{\GG_0 , \GG_1 , B \Larr A , \GD , \GL_0 , \GL_1 \vd C}{
  %       \deduce{\GD \vd A}{f'}
  %       &
  %       \infer[\cut]{\GG_0 , \GG_1 , B , \GL_0 , \GL_1 \vd C}{
  %         \deduce{\GG_1 , B , \GL_0 \vd C}{f''}
  %         &
  %         \deduce{\GG_0 , D, \GL_1 \vd C}{g}
  %       }
  %     }
  %   }
  % \end{array}
  % \]
    \[
  \begin{array}{l}
    \multicolumn{1}{c}{
      \proofbox{
      \infer[\cut]{\GG , A , \GL \vd C}{
        \infer[\ax]{A \vd A}{}
        &
        \deduce{\GG , A , \GL \vd C}{g}
      }
    }
    \quad\circeq\quad
    \proofbox{
      \deduce{\GG , A , \GL \vd C}{g}
    }
    }
    \\[5pt]
    \proofbox{
      \infer[\cut]{\GG , \GD_0 , A \ot B , \GD_1 , \GL \vd C}{
        \infer[\tl]{\GD_0 , A \ot B , \GD_1 \vd A'}{
          \deduce{\GD_0 , A , B , \GD_1 \vd A'}{f}
        }
        &
        \deduce{\GG , A' , \GL \vd C}{g}
      }
    }
    \\
    \hspace*{4cm}
    \circeq\quad
    \proofbox{
      \infer[\tl]{\GG , \GD_0 , A \ot B , \GD_1 , \GL \vd C}{
      \infer[\cut]{\GG , \GD_0 , A , B , \GD_1 , \GL \vd C}{
        \deduce{\GD_0 , A , B , \GD_1 \vd A'}{f}
        &
        \deduce{\GG , A' , \GL \vd C}{g}
      }
    }
    }
    \\[5pt]
    \proofbox{
      \infer[\cut]{\GG_0 , \GG_1 , \GD , A \Rarr B , \GL_0 , \GL_1 \vd C}{
        \infer[\Rl]{\GG_1 , \GD , A \Rarr B , \GL_0 \vd A'}{
          \deduce{\GD \vd A}{f'}
          &
          \deduce{\GG_1 , B , \GL_0 \vd C}{f''}
        }
        &
        \deduce{\GG_0 , A', \GL_1 \vd C}{g}
      }
    }
    \\
    \hspace*{4cm}
    \circeq\quad
    \proofbox{
      \infer[\Rl]{\GG_0 , \GG_1 , \GD , A \Rarr B , \GL_0 , \GL_1 \vd C}{
        \deduce{\GD \vd A}{f'}
        &
        \infer[\cut]{\GG_0 , \GG_1 , B , \GL_0 , \GL_1 \vd C}{
          \deduce{\GG_1 , B , \GL_0 \vd C}{f''}
          &
          \deduce{\GG_0 , A', \GL_1 \vd C}{g}
        }
      }
    }
  \end{array}
  \]
  \[
  \begin{array}{l}
    \proofbox{
      \infer[\cut]{\GG_0 , \GG_1 , B \Larr A , \GD , \GL_0 , \GL_1 \vd C}{
        \infer[\Ll]{\GG_1 , B \Larr A , \GD , \GL_0 \vd A'}{
          \deduce{\GD \vd A}{f'}
          &
          \deduce{\GG_1 , B , \GL_0 \vd C}{f''}
        }
        &
        \deduce{\GG_0 , A', \GL_1 \vd C}{g}
      }
    }
    \\
    \hspace*{4cm}
    \circeq\quad
    \proofbox{
      \infer[\Ll]{\GG_0 , \GG_1 , B \Larr A , \GD , \GL_0 , \GL_1 \vd C}{
        \deduce{\GD \vd A}{f'}
        &
        \infer[\cut]{\GG_0 , \GG_1 , B , \GL_0 , \GL_1 \vd C}{
          \deduce{\GG_1 , B , \GL_0 \vd C}{f''}
          &
          \deduce{\GG_0 , D, \GL_1 \vd C}{g}
        }
      }
    }
  \end{array}
  \]
\end{lemma}
\begin{proof}
  All proofs proceed by structural induction on the derivation $g$, following the cut admissibility procedure in the proof of Theorem \ref{thm:cut}.
\end{proof}
\begin{lemma}\label{lem:cut:cong}
  For any derivations $f$, $f'$, $g$ and $g'$, if $f \circeq f'$ and $g \circeq g'$, then $\cut \ (f ,g) \circeq \cut \ (f' , g')$.
\end{lemma}
\begin{proof}
  Proof proceeds by induction on all possible equations in \L.
\end{proof}
% Figure \ref{fig:conv} contains the examples of the equations of derivations in \L.
% The equations split to four groups: ($i$) $\eta$-expansion of $\ax$, i.e.\ an instance of $\ax$ of a compound formula is equivalent to the expansion with left and right rules applying on instances of $\ax$ of simpler formulae ($\ax$ of $A \Rarr B$ expands to $\Rr \ (\Rl \ (\ax ,\ax))$ with $\ax$ of $A$ and $\ax$ of $B$); ($ii$) left and right introduction rules can always be permuted ($\tr$ permutes with $\Rl$ in the figure); ($iii$) consecutive application of left rules is associative (two $\Rl$ instances permute with each other in the figure, where $B$ and $B'$ are in different sequents); ($iv$) consecutive application of left rules is commutative (two $\Rl$ instances permute with each other in the figure, where $B$ and $B'$ are in the same sequent).
% \vspace*{-0.8cm}
% also cut can be permuted with left rules in the first premise


\section{Proof-relevant Interpolation}
So far, Maehara interpolation only tells us about the \emph{existence} of the interpolant formula and corresponding derivations.
In this section, we are interested in the interaction between the interpolation procedure, derivations, and equivalence of derivations.
A similar question for intuitionistic logic with disjunction has been considered by \cub in the 90s \cite{vcubric1993results,Cubric1994}. Recently Saurin and his colleagues studied proof-relevant interpolation for (extensions) of classical linear logic \cite{Saurin2024} and generalize to non-cut-free sequent calculi and proof net \cite{Saurin2025}.
Veltri and Wan also studied proof-relevant interpolation for a skew logic \cite{VW2025} and nonassociative Lambek calculus \cite{VW2025Tableaux}.

\begin{theorem}[Proof-relevant interpolation]
  For derivations $g : \GG , D , \GL \vd C$ and $h : \GD \vd D$ obtained by interpolating the derivation $f : \GG , \GD , \GL \vd C$, $\cut \ (h , g) \circeq f$.
\end{theorem}
\begin{proof}
  The proof proceeds by structural induction on the derivation $f$.
  We show the proofs of the interesting subcases of $f = \Rl \ (f' , f'')$ mentioned in the proof of Theorem \ref{thm:MIP}.

    The first case is when $\GD$ splits both $\GG'$ and $\GD'$ where the interpolant formula is $E \ot D$. Recall the derivations $g = \tl \ (\Rl \ (g' , g''))$ and $h = \tr \ (h'' , h')$, therefore the goal is to show $\cut \ (\tr \ (h'' , h') , \tl \ (\Rl \ (g' , g''))) \circeq \Rl \ (f' , f'')$, which is witnessed by the following:
    \[
    \begin{array}{lr}
      \cut \ (\tr \ (h'' , h') , \tl \ (\Rl \ (g' , g''))) &
      \\
      = \cut \ (h'' , \cut \ (h' , \Rl \ (g' , g''))) & \text{(by definition of $\cut$)}
      \\
      = \Rl \ (\cut \ (h' , g') , \cut \ (h'' , g'')) & \text{(by definition of $\cut$, twice)}
      \\
      \circeq \Rl \ (f' , f'') & \text{(by ind. hyp. on $f'$ and $f''$}
      \\
      & \text{and congruence)}
    \end{array}
    \]
    The second case is when $\GD$ splits both $\GD'$ and $\GL'$ and contains $A \Rarr B$ and the interpolant formula is $D \Rarr E$.
    Recall the derivations $g = \Rr \ (\Rl \ (g' , h''))$ and $h = \Rl \ (h' , g'')$, therefore the goal is to show $$\cut \ (\Rr \ (\Rl \ (g' , h'')), \Rl \ (h' , g'')) \circeq \Rl \ (f' , f''),$$ which is witnessed by the following:
    \[
    \begin{array}{lr}
      \cut \ (\Rr \ (\Rl \ (g' , h'')), \Rl \ (h' , g'')) &
      \\
      = \cut \ (h' , \cut \ (\Rl \ (g' , h'') , g'')) & \text{(by definition of $\cut$)}
      \\
      \circeq \cut \ (h' , \Rl \ (g' , \cut \ (h'' , g''))) & \text{(by Lemma \ref{lem:left:rules:cut})}
      \\
      = \Rl \ (\cut \ (h' , g') , \cut \ (h'' , g'')) & \text{(by definition of $\cut$)}
      \\
      \circeq \Rl \ (f' , f'') & \text{(by ind. hyp. on $f'$ and $f''$}
      \\
      & \text{and congruence)}
    \end{array}
    \]
    Other cases are similar.
\end{proof}
% Veltri and Wan showed that Maehara interpolation procedure is well-defined wrt. equivalence relation on derivations in nonassociative Lambek calculus \cite{VW2025Tableaux}.
% In particular, consider two interpolant triples $(D , g , h)$ and $(D' , g' , h')$ which are obtained by Maehara interpolation on two derivations $f$ and $f'$, respectively. If $f \circeq f'$, then $D = D'$, $g \circeq g'$ and $h \circeq h'$.
In nonassociative Lambek calculus (\NL) two interpolant triples $(D , g , h)$ and $(D' , g' , h')$ are considered to equivalent if $D = D'$, $g \circeq g'$ and $h \circeq h'$. 
Veltri and Wan showed that Maehara interpolation procedure is well-defined wrt. equivalence relation on derivations in \NL\ \cite{VW2025Tableaux}, i.e. if two derivations a $\circeq$-related, then their interpolant triples are equivalent.

However, this is not the case for \L. Consider the permutative conversion between $\tr$ and $\Rl$ (the second item in Figure \ref{fig:conv}).
If the interpolant context splits $\GD$ and $\GO$, i.e. $\GD = \GD_0 , \GD_1$, $\GO = \GO_0 , \GO_1$ and the interpolant context of the endsequent is $\GD_1 , A \Rarr B , \GL , \GO_0$ (the red text traces the flow of the interpolant context in the derivation):
\begin{equation}\label{eq:ct:example}
  \begin{array}{l}
    \proofbox{\infer[\tr]{\GG , \GD_0 , \tred{\GD_1 , A \Rarr B , \GL , \GO_0} , \GO_1 \vd A' \ot B'}{
    \infer[\Rl]{\GG , \GD_0 , \tred{\GD_1 , A \Rarr B , \GL} \vd A'}{
      \deduce{\GD_0 , \tred{\GD_1} \vd \tred{A}}{f}
      &
      \deduce{\GG , \tred{B , \GL} \vd A'}{f'}
    }
    &
    \deduce{\tred{\GO_0} , \GO_1 \vd B'}{f''}
  }
  }
  \\
  \hspace*{2cm}\circeq\qquad
  \proofbox{\infer[\Rl]{\GG , \GD_0 , \tred{\GD_1 , A \Rarr B , \GL , \GO_0} , \GO_1 \vd A' \ot B'}{
    \deduce{\GD_0 , \tred{\GD_1} \vd \tred{A}}{f}
    &
    \infer[\tr]{\GG , \tred{B , \GL , \GO_0 }, \GO_1 \vd A' \ot B'}{
      \deduce{\GG , \tred{B , \GL} \vd A'}{f'}
      &
      \deduce{\tred{\GO_0} , \GO_1 \vd B'}{f''}
    }
  }
  }
  \end{array}
\end{equation}
By running the interpolation on derivations in (\ref{eq:ct:example}), we first obtain the following interpolant triples for $f$, $f'$ and $f''$, respectively:
\[
\begin{array}{l}
(D , g : D , \GD_1 \vd A , h : \GD_0 \vd D)
\\[3pt]
 (E , g' : \GG , E \vd A' , h' : B , \GL \vd E)
\\[3pt]
 (F , g'' : F , \GO_1 \vd B' , h'' : \GO_0 \vd F)
%  (D ,& g : D , \GD_1 \vd A ,& h : \GD_0 \vd D)
% \\[3pt]
%  (E ,& g' : \GG , E \vd A' ,& h' : B , \GL \vd E)
% \\[3pt]
%  (F ,& g'' : F , \GO_1 \vd B' ,& h'' : \GO_0 \vd F)
% & f' 
% & f''
% \\
% \hline
% D & E & F
% \\
% \hline
% g : D , \GD_1 \vd A & g' : \GG , E \vd A' & g'' : F , \GO_1 \vd B'
% \\
% \hline
% h : \GD_0 \vd D & h' : B , \GL \vd E & h'' : \GO_0 \vd F
\end{array}
\]
Then the interpolant triples of the derivations in (\ref{eq:ct:example}) are:
\[
\begin{array}{lr}
\arraycolsep=20pt
  % \begin{array}{c}
  %   (D \Rarr E) \ot F
  %   \\
  %   \tl \ (\tr \ (\Rl \ (h , g') , g'')) : \GG , \GD_0 , (D \Rarr E) \ot F , \GO_1 \vd A' \ot B'
  %   \\
  %   \tr\ (\Rr \ (\Rl \ (g , h')), h'') : \GD_1 , A \Rarr B , \GL , \GO_0 \vd (D \Rarr E) \ot F
  % \end{array}
  % \\[0.5cm]
  % \begin{array}{c}
  %   D \Rarr (E \ot F)
  %   \\
  %   \Rl \ (h , \tl \ (\tr \ (g' , g''))) : \GG , \GD_0 , D \Rarr (E \ot F) , \GO_1 \vd A' \ot B'
  %   \\
  %   \Rr \ (\Rl \ (g , \tr \ (h' , h''))) : \GD_1 , A \Rarr B , \GL , \GO_0 \vd D \Rarr (E \ot F)
  % \end{array}
  \begin{array}{c}
    \text{(for the top derivation)}
    \\
    (D \Rarr E) \ot F
    \\
    \tl \ (\tr \ (\Rl \ (h , g') , g'')) 
    \\
    \tr\ (\Rr \ (\Rl \ (g , h')), h'')
  \end{array}
  &
  \begin{array}{c}
    \text{(for the bottom derivation)}
    \\
    D \Rarr (E \ot F)
    \\
    \Rl \ (h , \tl \ (\tr \ (g' , g'')))
    \\
    \Rr \ (\Rl \ (g , \tr \ (h' , h'')))
  \end{array}
\end{array}
\]
% two interpolant formulae, $(D \Rarr E) \ot F$ for the top derivation and $D \Rarr (E \ot F)$ for the bottom one, given that $D$, $E$ and $F$ are interpolant formulae from $f$, $f'$, and $f''$, respectively.
The two interpolant formulae are not logically equivalent and in fact there is only way of derivability (notice that the derivation is not unique):
\begin{equation}\label{eq:t:example}
  \proofbox{
    \infer[\Rr]{(D \Rarr E) \ot F \vd D \Rarr (E \ot F)}{
    \infer[\tl]{D , (D \Rarr E) \ot F \vd E \ot F}{
      \infer[\tr]{D , D \Rarr E , F \vd E \ot F}{
        \infer[\Rl]{D , D \Rarr E \vd E}{
          \infer[\ax]{D \vd D}{}
          &
          \infer[\ax]{E \vd E}{}
        }
        &
        \infer[\ax]{F \vd F}{}
      }
    }
  }
  }
\end{equation}
This example leads to a relaxed notion of equivalence on interpolant triples.
\begin{definition}
  Two interpolant triples $(D , g , h)$ and $(D' , g' , h')$ are equivalent, denoted as $(D , g , h) \sim (D' , g' , h')$, if
  \begin{itemize}
    \item there exist a derivation $t : D \vd D'$, such that $g \circeq \cut \ (t , g')$ and $\cut \ (h , t) \circeq h'$ or
    \item there exist a derivation $t : D' \vd D$, such that $g' \circeq \cut \ (t , g)$ and $\cut \ (h' , t) \circeq h$.
  \end{itemize}
\end{definition}
Intuitively, two interpolant triples are considered to be equivalent if there exists a proof $t$ between the two interpolant formulae and we obtain equivalent derivations when we substitute one interpolant formula for the other appropriately.

Let us check whether the interpolant triples in the example above are equivalent.
The derivation (\ref{eq:t:example}) is a witness of the sequent $(D \Rarr E) \ot F \vd D \Rarr (E \ot F)$.
We then check the equivalence of derivations.
The first case is:
\[
  \begin{array}{lr}
    \cut \ (\Rr \ (\tl \ (\tr \ (\Rl \ (\ax , \ax) , \ax))), \Rl \ (h , \tl \ (\tr \ (g' , g'')))) &
    \\
    = \cut \ (h , \cut \ (\tl \ (\tr \ (\Rl \ (\ax , \ax) , \ax)) , \tl \ (\tr \ (g' , g'')))) &
    \\ & \hspace*{-2cm}\text{(by definition of $\cut$)}
    \\
    = \cut \ (h , \tl \ (\cut \ (\tr \ (\Rl \ (\ax , \ax) , \ax) , \tl \ (\tr \ (g' , g''))))) &
    \\ & \hspace*{-2cm}\text{(by definition of $\cut$)}
    \\
    = \cut \ (h , \tl \ (\cut \ (\Rl \ (\ax , \ax), \cut \ (\ax , \tr \ (g' , g''))))) &
    \\ & \hspace*{-2cm}\text{(by definition of $\cut$)}
    \\
    \circeq \cut \ (h , \tl \ (\cut \ (\Rl \ (\ax , \ax), \tr \ (g' , g'')))) &
     \hspace*{-2cm}\text{(by Lemma \ref{lem:left:rules:cut})}
    \\
    = \cut \ (h , \tl \ (\tr \ (\cut \ (\Rl \ (\ax , \ax) , g') , g''))) &
     \hspace*{-2cm}\text{(by definition of $\cut$)}
    \\
    \circeq \cut \ (h , \tl \ (\tr \ (\Rl \ (\ax , \cut \ (\ax , g')) , g''))) &
     \hspace*{-2cm}\text{(by Lemma \ref{lem:left:rules:cut})}
    \\
    \circeq \cut \ (h , \tl \ (\tr \ (\Rl \ (\ax , g') , g''))) &
     \hspace*{-2cm}\text{(by Lemma \ref{lem:left:rules:cut})}
    \\
    = \tl \ (\tr \ (\Rl \ (\cut \ (h , \ax) , g'), g'')) &
     \hspace*{-2cm}\text{(by definition of $\cut$)}
    \\
    = \tl \ (\tr \ (\Rl \ (h , g'), g'')) &
     \hspace*{-2cm}\text{(by definition of $\cut$)}
  \end{array}
\]
The second case is:
\[
  \begin{array}{lr}
    \cut \ (\tr \ (\Rr \ (\Rl \ (g , h')) , h'') , \Rr \ (\tl \ (\tr \ (\Rl \ (\ax , \ax) , \ax)))) &
    % \\ & \hspace*{-2cm}\text{(by definition of $\cut$)}
    \\
    = \Rr \ (\cut \ (\tr \ (\Rr \ (\Rl \ (g , h')) , h'')  , \tl \ (\tr \ (\Rl \ (\ax , \ax) , \ax)))) &
    \\ & \hspace*{-2cm}\text{(by definition of $\cut$)}
    \\
    = \Rr \ (\cut \ (\Rr \ (\Rl \ (g , h')) , \cut \ (h'' , \tr \ (\Rl \ (\ax , \ax), \ax)))) &
    \\ & \hspace*{-2cm}\text{(by definition of $\cut$)}
    \\
    = \Rr \ (\cut \ (\Rr \ (\Rl \ (g , h')) , \tr \ (\cut \ (\Rl \ (\ax , \ax) , \cut \ (h'' , \ax))))) &
    \\ & \hspace*{-2cm}\text{(by definition of $\cut$)}
    \\
    = \Rr \ (\cut \ (\Rr \ (\Rl \ (g , h')) , \tr \ (\cut \ (\Rl \ (\ax , \ax) , h'')))) &
    \\ & \hspace*{-2cm}\text{(by definition of $\cut$)}
    \\
    = \Rr \ (\tr \ (\cut \ (\Rr \ (\Rl \ (g , h')) , \Rl \ (\ax , \ax)), h'')) &
    \\ & \hspace*{-2cm}\text{(by definition of $\cut$)}
    \\
    = \Rr \ (\tr \ (\cut \ (\ax , \cut \ (\Rl \ (g , h') , \ax)) , h'')) &
    \hspace*{-2cm}\text{(by definition of $\cut$)}
    \\
    =\Rr \ (\tr \ (\Rl \ (g , h')) , h'') &
    \hspace*{-4cm}\text{(by definition of $\cut$ and Lemma \ref{lem:left:rules:cut})}
    \\
    \circeq \Rr \ (\Rl \ (g , \tr \ (h' , h''))) & \hspace*{-2cm}\text{(by definition of $\circeq$)}
  \end{array}
\]
\begin{theorem}\label{thm:welldef:L}
  For any two derivations $f , f' : \GG , \GD , \GL \vd C$ and their interpolant triples $(D , g , h)$ and $(D' , g' , h')$, if $f \circeq f'$, then $(D , g , h) \sim (D' , g' , h')$.
\end{theorem}
\begin{proof}
  Proof proceeds by examining all equations in the equivalence relation $\circeq$.
  We show the case of the equation that two instances of $\Rl$ are commutative (the last item in Figure \ref{fig:conv}) while other cases are similar.
  The cases are divided into three groups:
  \begin{enumerate}
    \item The interpolant context is ($i$) contained in $\GG$, $\GD_0$, $\GL$, $\GD_1$, or $\GO$, ($ii$) it contains $\GD_0 , A\Rarr B$ but not $\GD_1 , A' \Rarr B'$ and the other way around or ($iii$) it contains $\GD_0 , A \Rarr B , \GL , \GD_1 , A' \Rarr B'$. 
    % \begin{itemize}
    %   \item contained in $\GG$, $\GD_0$, $\GL$, $\GD_1$, or $\GO$ or
    %   \item it contains $\GD_0 , A\Rarr B$ but not $\GD_1 , A' \Rarr B'$ and the other way around or
    %   \item it contains $\GD_0 , A \Rarr B , \GL , \GD_1 , A' \Rarr B'$. 
    % \end{itemize}
    The interpolant triples of the two derivations in the equation are dictated by one of the derivations in $f$, $f'$ and $f''$. The triples will have the same interpolant formulae and the equivalence of derivations are witnessed by the equations in $\circeq$.
    For example, if the interpolant context is contained in $\GD_0$, then the interpolant triples are $(D , \Rl \ (g,\Rl\ (f' , f'')) , h)$ and $(D , \Rl \ (f' , \Rl \ (g , f'')) , h)$ where $(D , g ,h)$ is the interpolant triple of $f$. The interpolant formulae and derivation $h$ are the same and $\Rl \ (g,\Rl\ (f' , f'')) \circeq \Rl \ (f' , \Rl \ (g , f''))$ is an instance of the commutativity of $\Rl$. Therefore, the two interpolant triples are equivalent. 
    \item The interpolant context splits ($i$) $\GG$ and $\GD_0$, ($ii$) $\GG$ and $\GD_1$, ($iii$) $\GD_0$ and $\GL$, ($iv$) $\GD_0$ and $\GO$, ($v$) $\GL$ and $\GD_1$, or ($vi$) $\GD_1$ and $\GO$.
    % \begin{itemize}
    %   \item $\GG$ and $\GD_0$, %(tensor of $f$ and $f''$), 
    %   \item $\GG$ and $\GD_1$, %($\Rarr$ of $f'$ and $f''$),
    %   \item $\GD_0$ and $\GO$, %($\Rarr$ of $f$ and $f''$),
    %   \item $\GL$ and $\GD_1$, or %(tensor of $f'$ and $f''$) or
    %   \item $\GD_1$ and $\GO$. %($\Rarr$ of $f'$ and $f''$)
    % \end{itemize}
    In the these cases, the interpolant triples of the two derivations in the equation is dictated by two of the derivations in $f$, $f'$ and $f''$.
    Again, the triples will have the same interpolant formulae and the equivalence of derivations are witnessed by the equations in $\circeq$.
    For example, if the interpolant context splits $\GG$ and $\GD_1$, then the interpolant triples of the derivations in the equation are dictated only by $f'$ and $f''$. Assume that the interpolant triples of $f'$ and $f''$ are $(E , g' , h')$ and $(F , g'' , h'')$, respectively.
    The interpolant triples of derivations in the equation are:
    \[
    \begin{array}{l}
      (F \ot E , \tl \ (\Rl \ (g' , g'')) , \Rl \ (f , \tr\ (h' , h''))) 
      \\
      % \text{and}
      % \\
      (F \ot E , \tl \ (\Rl \ (g' , g'')) , \tr \ (h' , \Rl \ (f , h'')))
    \end{array}
    \] 
    % The interpolant formulae and the derivation $\tl \ (\Rl \ (g' , g''))$ are the same and 
    The equation $\Rl \ (f , \tr\ (h' , h'')) \circeq \tr \ (h' , \Rl \ (f , h''))$ is an instance of permutation between $\Rl$ and the second premise of $\tr$. Therefore, the two interpolant triples are equivalent.
    % splits $\GG$ and $\GD_0$ (tensor of $f$ and $f''$), splits $\GD_0$ and $\GL$ ($\Rarr$ of $f$ and $f''$), splits $\GD_0$ and $\GO$, ($\Rarr$ of $f$ and $f''$), splits $\GL$ and $\GD_1$ (tensor of $f'$ and $f''$), splits $\GD_1$ and $\GO$ ($\Rarr$ of $f'$ and $f''$)
    \item The last one is the interpolant context splits $\GD_0$ and $\GD_1$, i.e. the interpolant context is $\GD''_0 , A \Rarr B , \GL , \GD'_1$ where $\GD_0 = \GD'_0 , \GD''_0$, $\GD_1 = \GD'_1 , \GD''_1$. Similar to the case for (\ref{eq:ct:example}), we first run the interpolation procedure to the left and right derivations in the equation to obtain the following data:
    \[
      \begin{array}{rl}
      \begin{array}{r}
        \text{(interpolant triples}
        \\
        \text{of $f$, $f'$, and $f''$)}
      \end{array}
      &
      \begin{array}{l}
        (D , g : D , \GD''_0 \vd A , h : \GD'_0 \vd D)
      \\[3pt]
      (E , g' : E , \GD''_1 \vd A' , h' : \GD'_1 \vd E)
      \\[3pt]
      (F , g'' : \GG , F , B' , \GO \vd C , h'' : B , \GL \vd F)
      \end{array}
      % & f' 
      % & f''
      % \\
      % \hline
      % D & E & F
      % \\
      % \hline
      % g : D , \GD_1 \vd A & g' : \GG , E \vd A' & g'' : F , \GO_1 \vd B'
      % \\
      % \hline
      % h : \GD_0 \vd D & h' : B , \GL \vd E & h'' : \GO_0 \vd F
      \end{array}
    \]
    \[
      \begin{array}{lr}
      \arraycolsep=20pt
        \begin{array}{c}
          \text{(for the top derivation)}
          \\
          D \Rarr (F \ot E)
          \\
          \Rl (h , \tl \ (\Rl \ (g' , g'')))
          \\
          \Rr \ (\Rl \ (g , (\tr \ (h'' , h'))))
        \end{array}
        &
        \begin{array}{c}
          \text{(for the bottom derivation)}
          \\
          (D \Rarr F) \ot E
          \\
          \tl \ (\Rl \ (g' , \Rl (h , g'')))
          \\
          \tr \ (\Rr \ (\Rl \ (g , h'')), h')
        \end{array}
      \end{array}
    \]
    The two interpolant formulae are not logically equivalent but there exists a derivation $\Rr \ (\tl \ (\tr \ (\Rl \ (\ax , \ax),\ax))) : (D \Rarr F) \ot E \vd D \Rarr (F \ot E)$ such that the following two equation hold:
    % We then check the equivalence of derivations. The first case is:
    % \[
    %   \begin{array}{lr}
    %     \cut \ (\Rr \ (\tl \ (\tr \ (\Rl \ (\ax , \ax),\ax))) , \Rl (h , \tl \ (\Rl \ (g' , g'')))) &
    %     \\
    %     hi
    %   \end{array}
    % \]
    \[
    \begin{array}{l}
      \tl \ (\Rl \ (g' , \Rl (h , g''))) 
      \\
      \hspace*{1.4cm} \circeq \cut \ (\Rr \ (\tl \ (\tr \ (\Rl \ (\ax , \ax),\ax))) , \Rl (h , \tl \ (\Rl \ (g' , g''))))
      \\[5pt]
      \cut\ (\tr \ (\Rr \ (\Rl \ (g , h'')), h') , \Rr \ (\tl \ (\tr \ (\Rl \ (\ax , \ax),\ax)))) 
      \\
      \hspace*{1.4cm} \circeq  \Rr \ (\Rl \ (g , (\tr \ (h'' , h'))))
    \end{array}
    \]
    The chains of reasoning are similar to the ones in the proof of equivalence of interpolant triples of (\ref{eq:ct:example}).
  \end{enumerate} 
\end{proof}
\section{Formalization in Agda}
As the readers might have noticed, the proof of Theorem \ref{thm:welldef:L} requires enormous amount of case distinction. It makes the proof error-prone and demanding to human. Therefore we formalize the $\Rarr,\ot$ fragment of \L\ in the Agda proof assistant as an experiment to test our calculation on pen and paper.

The formalization of the syntax of the $\Rarr,\ot$ fragment of \L\ is standard. We define the formulae, sequents as inductive types with the appropriate constructors for each connective and its corresponding left and right rules.
The equivalence relation of derivations $\circeq$ is defined as an inductive datatype including the constructors that makes $\circeq$ as an equivalence and congruence relation and all permutative conversions.
% The functions of cut admissibility and Maehara interpolation are constructed similarly as Veltri and Wan did for the nonassociative Lambek calculus \cite{VW2025Tableaux}. 
For Maehara interpolation, we first declare a record type \MIP:
% Therefore we formalize the $\Rarr,\ot$ fragment of \L\ and verify its cut admissibility, Maehara interpolation, proof-relevant interpolation and well-definedness of interpolation.
\[
\begin{array}{rl}
  \multicolumn{2}{l}{\record \:\:  \MIP ~ (\GG ~ \GD ~ \GL : \mf{Cxt}) ~ (C : \Fma) : \Set \:\: \where} \\
  \multicolumn{2}{l}{\quad \mf{constructor} ~ \mf{intrp}} \\
  \multicolumn{2}{l}{\quad \field} \\
  \;\; \quad D &: \Fma \\[2pt]
  \;\; \quad g &: \GG , D , \GL \vd C \\[2pt]
  \;\; \quad h &: \GD \vd D 
\end{array}
\]
Elements of the type $\MIP ~ \GG ~ \GD ~ \GL ~ C$ are triples consisting of a formula $D$ and two derivations $g$ and $h$, i.e.\ the interpolant triples.

Maehara interpolation is then witnessed by the construction of the following function $\mf{mip}$:
\[
\begin{array}{l}
\mathsf{mip} : (f : \GO \vd C) ~ (eq : \GO \equiv \GG \conc \GD \conc \GL) \to \MIP ~\GG ~ \GD ~ \GL ~C
\end{array}
\]
The function is constructed by pattern-matching on the derivation $f$.

The equivalence of interpolant triples is implemented in two steps.
The first step is to define an implication between two interpolant triples as a dependent product type:
\[
  \begin{array}{l}
  \mathsf\_{\Vdash}\_ : \forall \{\GG ~ \GD ~ \GD ~ C\} \to (n ~ n' : \MIP \GG ~ \GD ~ \GL ~C) \to \Set
  \\
  \mathsf\_{\Vdash}\_ ~ \{ \GG \} ~ (\mf{intrp} ~ D ~ g ~ h) ~ (\mf{intrp} ~ D' ~ g' ~ h')
  \\
  \hspace*{2cm}= \Sigma ~ (D \vd D') ~ \Gl ~ t ~ \to (g \circeq \cut ~ t ~ g') \times (\cut ~ h ~ t \circeq h') 
  \end{array}
\]
The elements of this dependent product are a derivation $t$ between interpolant formulae and the equivalence of the pairs of derivations cut with $t$.

Second, the equivalence relation of interpolant triples is implemented as an inductive type:
\[
\begin{array}{rl}
  \multicolumn{2}{l}{\data \:\:  \_{\sim}\_ ~ \{\GG ~ \GD ~ \GL~ C\} : \MIP~ \GG ~ \GD ~ \GL~ C\to \MIP~ \GG ~ \GD ~ \GL~ C \to \Set \:\: \where} \\
  \;\;\; \mf{refl} &: \{n : \MIP \GG ~ \GD ~ \GL ~C\} \to n \sim n \\
  \;\;\; \rcons &: \{n~n'~p : \MIP \GG ~ \GD ~ \GL ~C\} \to n \Vd n' \to n' \sim p \to n \sim p \\
  \;\;\; \lcons &: \{n~n'~p : \MIP \GG ~ \GD ~ \GL ~C\} \to n' \Vd n \to n' \sim p \to n \sim p
\end{array}
\]
The constructor $\mf{refl}$ means that any interpolant triple is equivalent to itself. 
The other two means that for any three interpolant triples $n$, $n'$, and $p$, if $n' \sim p$, then we can update the equivalence to $n \sim p$ whenever we have a proof of $n \Vd n'$ or $n' \Vd n$.

Symmetry and transitivity of the relation $\_{\sim}\_$ are derivable so they are not included in the datatype.

The well-definedness of Maehara interpolation wrt.\ $\circeq$ is implemented as a record type:
\[
\begin{array}{rl}
  \multicolumn{2}{l}{\record \:\:  \MIP{\circeq} ~ (\GG ~ \GD ~ \GL : \mf{Cxt}) ~ (C : \Fma) ~ (n ~ n' : \MIP \GG ~ \GD ~ \GL ~ C) : \Set \:\: \where} \\
  \multicolumn{2}{l}{\quad \mf{constructor} ~ \mf{intrp{\circeq}}} \\
  \multicolumn{2}{l}{\quad \field} \\
  \;\; \quad eq &: n \sim n'
\end{array}
\]
It has only one element, the equivalence of two interpolant triples.
Well-definedness of Maehara interpolation wrt.\ $\circeq$ is then witnessed by the construction of the following function $\mf{mip}{\circeq}$:
\[
\begin{array}{l}
\mathsf{mip}{\circeq} : \forall \GG ~ \GD ~ \GL ~\{\GO ~ C\} \to \{f ~ f' : \GO \vd C\} 
\\
\hspace*{0.4cm}\to (eq : \GO \equiv \GG \conc \GD \conc \GL) 
\to (p : f \circeq f') 
\\
\hspace*{0.4cm}\to \MIP{\circeq} ~\GG ~ \GD ~ \GL ~C ~(\mf{mip} ~ \GG ~ \GD ~ \GL ~f~ eq) ~ (\mf{mip} ~ \GG ~ \GD ~ \GL ~f'~ eq)
\end{array}
\]
The function is constructed by pattern-matching on $p$, the proof of equivalence of the derivations $f$ and $f'$. 

\underline{N. B.} 
The formulation postulates the admissible equations in Lemmata \ref{lem:left:rules:cut} and \ref{lem:cut:cong}. These equations are provable in Agda but we chose to postulate them and leave for the future.
We also include the multiplicative unit $\I$ and its left and right introduction rules to work with the sequents with lists of formulae as their antecedents. In this setting, we can smoothly prove Maehara interpolation following exactly the same strategy in the proof of Theorem \ref{thm:MIP} without dealing with with non-empty list in Agda. The equations of derivations related to $\I$ are not included as well. The variable condition is also not considered in the formalization.

\section{Proof-relevant Maehara Multi-Interpolation of Product-free Lambek Calculus}
In this section, we show that the product-free Lambek calculus (\LPfree) also enjoys proof-relevant interpolation and its Maehara interpolation procedure is well-defined wrt. $\circeq$.

The formulae of \LPfree\ are obtained by excluding the formulae of the shape $A \ot B$ from the set of formulae of \L.
The derivations in \LPfree\ are generated inductively by the rules in (\ref{eq:seqcalc}) except $\tl$ and $\tr$.
\begin{theorem}[Cut admissibility]\label{thm:cut:LPfree}
  The cut rule
  \[
  \infer[\cut]{\GG , \GD , \GL \vd C}{
    \deduce{\GD \vd A}{f}
    &
    \deduce{\GG , A , \GL \vd C}{g}
  }
  \]
   is admissible in \LPfree.
\end{theorem}
We introduce a few admissible rules that will be used later in the section. 
First, given a list of formulae $\GD = A_1, \dots, A_n$, we define an iterated version of the rule $\Rr$.
Below and in the rest of the section, we write $\GD \Rarr^* B$ for the formula $A_n \Rarr (A_{n-1} \Rarr (\dots (A_1 \Rarr B)\dots))$.
The double-lines denote the equality of sequents.
\begin{equation}
  \begin{array}{c}
    \proofbox{
      \infer[\Rsr]{\GG \vd \GD \Rs B}{
        \deduce{\GD , \GG \vd B}{f}
        }
    }
    \quad=
    \proofbox{
      \infer={\GG \vd \GD \Rs B}{
        \infer[\Rr]{\GG \vd A_n \Rarr (A_{n-1} \Rarr (\dots (A_1 \Rarr B)\dots))}{
          \deduce[\vdots]{A_n , \GG \vd A_{n-1} \Rarr (\dots (A_1 \Rarr B)\dots)}{
              \infer[\Rr]{A_2 , \dots , A_n , \GG \vd A_1 \Rarr B}{
                \infer={A_1 , A_2 , \dots , A_n , \GG \vd B}{
                  \deduce{\GD , \GG \vd B}{f}
                }
              }
          }
        }
      }
    }
  \end{array}
\end{equation} 
Second, given a list of derivations, $f_i : \GD_i \vd A_i$ for $i \in [1,\dots, n]$, we define an iterated version of $\Rl$ that has $n$ applications of $\Rl$:
\begin{equation}
  \begin{array}{l}
    \proofbox{
      \infer[\Rsl]{\GG, \GD_1 , \dots , \GD_n , \GD \Rs B,\GL \vd C}{
        \deduce{\GD_i \vd A_i}{[f_i]}
        &
        \deduce{\GG , B , \GL \vd C}{g}
      }
    }
    \\
    \hspace*{0.5cm}=\quad
    \proofbox{
      \infer={\GG, \GD_1 , \dots , \GD_n , \GD \Rs B,\GL \vd C}{
        \infer[\Rl]{\GG, \GD_1 , \dots , \GD_n , A_n \Rarr (A_{n-1} \Rarr (\dots (A_1 \Rarr B)\dots)),\GL \vd C}{
          \deduce{\GD_n \vd A_n}{f_n}
          &
          \deduce[\vdots]{\GG, \GD_1 , \dots , \GD_{n-1} , A_{n-1} \Rarr (\dots (A_1 \Rarr B)\dots),\GL \vd C}{
            \infer[\Rl]{\GG , \GD_1 , A_1 \Rarr B , \GL \vd C}{
              \deduce{\GD_1 \vd A_1}{f_1}
              &
              \deduce{\GG , B , \GL \vd C}{g}
            }
          }
        }
      }
    }
  \end{array}
\end{equation}
Both of the admissible rules can be defined for $\Larr$ in a similar manner. Given $\GD = A_1, \dots, A_n$, we write $B \Larr^* \GD$ for the formula $((\dots(B \Rarr A_n) \dots) \Larr A_2) \Larr A_1$.
The definitions of $\Lsr$ and $\Lsl$ are similar to $\Rsr$ and $\Rsl$ but the order of the formulae are reversed.
% Given the same list of formulae $\GD$, we can define an iterated version of $\Lr$ similarly as above.. We write $B \Larr^* \GD$ for the formula $((\dots(B \Rarr A_n) \dots) \Larr A_2) \Larr A_1$.
% \[
%   \proofbox{\infer[\Lsr]{\GG \vd B \Ls \GD}{
%     \deduce{\GG , \GD \vd B}{f}
%   }}
%   \quad=
%   \proofbox{
%     \infer={\GG \vd B \Ls \GD}{
%         \infer[\Lr]{\GG \vd ((\dots(B \Rarr A_n) \dots) \Larr A_2) \Larr A_1}{
%           \deduce[\vdots]{\GG, A_1 \vd (\dots(B \Rarr A_n) \dots) \Larr A_2}{
%               \infer[\Lr]{\GG, A_1 , \dots , A_{n-1} \vd B \Larr A_n}{
%                 \infer={\GG , A_1 , \dots , A_{n-1},A_n \vd B}{
%                   \deduce{\GG, \GD \vd B}{f}
%                 }
%               }
%           }
%         }
%       }
%   }
% \]
Next, given a list of derivations, $f_i : \GD_i \vd A_i$ for $i \in [1,\dots, n]$ and a derivation $g : \GG , A_1, \dots, A_n, \GL \vd C$, we define the iterated version of $\cut$ that contains $n$ applications of $\cut$ on distinct formulae in the context of $g$:
\begin{equation}
  \begin{array}{l}
    \proofbox{
      \infer[\cuts]{\GG , \GD_1 , \dots , \GD_n , \GL \vd C}{
        \deduce{\GD_i \vd A_i}{[f_i]}
        &
        \deduce{\GG , A_1, \dots, A_n, \GL \vd C}{g}
      }
    }
    \\
    \hspace*{3cm}=\quad
    \infer[\cut]{\GG , \GD_1 , \dots , \GD_n , \GL \vd C}{
      \deduce{\GD_1 \vd A_1}{f_1}
      &
      \deduce[\vdots]{\GG , A_1 , \GD_2 , \dots , \GD_n , \GL \vd C}{
        \infer[\cut]{\GG , A_1 , \dots , A_{n-1} , \GD_n , \GL \vd C}{
          \deduce{\GD_n \vd A_n}{f_n}
          &
          \deduce{\GG , A_1, \dots, A_n, \GL \vd C}{g}
        }
      }
    }
  \end{array}
\end{equation}
\begin{theorem}[Maehara multi-interpolation \cite{Pentus1997}]\label{thm:MIP:LPfree}
~
\\
  Given a derivation $f : \GG , \GD , \GL \vd C$, there exist a partition $\langle \GD_1, \dots, \GD_n\rangle$ of $\GD$, a list of formula $D_i$ for $i\in[1,\dots,n]$ a derivation $g : \GG, D_1,\dots,D_n, \GL \vdash C$ and a list of derivations $h_i : \GD_i \vdash D_i$ such that $\vars{D_1 , \dots, D_n} \subseteq \vars{\GD} \cap \vars{\GG, \GL, C}$.
\end{theorem}
\begin{proof}
  Proof proceeds similarly as the proof of Theorem \ref{thm:MIP}.
  Here we briefly discuss how the new proofs of the two interesting cases of $f = \Rl \ (f' , f'')$.
  The first case is when the interpolant context splits $\GG'$ and $\GD'$ where the interpolant formula was a tensor formula. 
  In this case, we apply inductive hypothesis on $f'$ and $f''$ respectively to obtain:
  \begin{itemize}
    \item a partition $\langle \GD'_{01}, \dots , \GD'_{0n}\rangle$ of $\GD'_0$ and a list of formulae $D_i$ for $i\in[1,\dots,n]$,
    \item a derivation $g' : D_1 , \dots ,D_n , \GD'_1 \vd A$, a list of derivations $h'_i : \GD'_{0i} \vdash D_i$ such that
    \item $\vars{D_1 , \dots, D_n} \subseteq \vars{\GD'_0} \cap \vars{\GD'_1, A}$
  \end{itemize}
  \begin{itemize}
    \item a partition $\langle \GG'_{11}, \dots , \GG'_{1m}\rangle$ of $\GG'_1$ and a list of formulae $E_j$ for $j\in[1,\dots,m]$,
    \item a derivation $g'' : \GG'_0 , E_1 , \dots ,E_m , B , \GL' \vd A$, a list of derivations $h''_j : \GG'_{1j} \vdash E_j$ such that
    \item $\vars{E_1 , \dots, E_m} \subseteq \vars{\GG'_1} \cap \vars{\GG'_0 , B , \GL' , C}$
  \end{itemize}
  The desired partition of $\GG'_1, \GD'_0$ is $\langle \GG'_{11}, \dots , \GG'_{1m}, \GD'_{01}, \dots , \GD'_{0n}\rangle$ and the list of interpolant formulae is $E_1 , \dots, E_m, D_1,\dots, D_n$. The desired derivations are \linebreak $\Rl \ (g' , g'')$ and $h''_1, \dots, h''_m, h'_1, \dots, h'_n$.

  The second case is when the interpolant context splits both $\GD'$ and $\GL'$ and contains $A \Rarr B$. By inductive hypothesis, we obtain:
  \begin{itemize}
    \item a partition $\langle \GD'_{01}, \dots , \GD'_{0n}\rangle$ of $\GD'_0$ and a list of formulae $D_i$ for $i\in[1,\dots,n]$,
    \item a derivation $g' : D_1 , \dots ,D_n , \GD'_1 \vd A$, a list of derivations $h'_i : \GD'_{0i} \vdash D_i$ such that
    \item $\vars{D_1 , \dots, D_n} \subseteq \vars{\GD'_0} \cap \vars{\GD'_1, A}$
  \end{itemize}
  \begin{itemize}
    \item a partition $\langle \GL'_{01}, \dots , \GL'_{0m}\rangle$ of $\GL'_0$ and a list of formulae $E_j$ for $j\in[1,\dots,m]$,
    \item a derivation $g'' : \GG' , E_1 , \dots ,E_m, \GL'_1 \vd A$, a derivation $h''_1 : B , \GL'_{01} \vd E_1$ and a list of derivations $h''_j : \GL'_{0j} \vdash E_j$ for $j\in[2,\dots,m]$ such that
    \item $\vars{E_1 , \dots, E_m} \subseteq \vars{\GG'_1} \cap \vars{\GG,  \GL'_1 , C}$
  \end{itemize}
  The desired partition of $\GD'_1 , A \Rarr B , \GL'_1$ is $\langle (\GD'_1, B , \GL'_{01}), \GL'_{02}, \dots \GL'_{0m}\rangle$ and the list of interpolant formulae is $\GO \Rs E_1 , E_2 , \dots , E_j$ where $\GO = D_1 , \dots , D_n$. The desired derivations are $\Rsl \ ([h_i] , g'')$ and $\Rsr \ (\Rl \ (g' , h''_1)), h''_2, \dots , h''_j$.
\end{proof}
\begin{definition}
  equivalence of triples
\end{definition}
\begin{theorem}[Proof-relevant interpolation]
~
\\
  For derivations $g : \GG , D_1, \dots, D_n , \GL \vd C$ and $h_i : \GD \vd D$ for $i \in [1 ,\dots, n]$ obtained by interpolating the derivation $f : \GG , \GD , \GL \vd C$, $\cuts \ ([h_i] , g) \circeq f$.
\end{theorem}
\begin{proof}
  Proof proceeds by induction on $f$ with the help of the admissible equations (not yet defined!)
\end{proof}
\begin{theorem}
  well-definedness
\end{theorem}
\[
calculus\ decorated\ by\ interpolant\ triples
\]


\section{Conclusions}
will the interpolant triples in complete calculi by \cite{Hetzl2024} are just collection of logically equivalent formulae?
\begin{credits}
\subsubsection{\ackname} This work was supported by the Estonian Research Council grant PSG749. 

% \subsubsection{\discintname}
% It is now necessary to declare any competing interests or to specifically
% state that the authors have no competing interests. Please place the
% statement with a bold run-in heading in small font size beneath the
% (optional) acknowledgments\footnote{If EquinOCS, our proceedings submission
% system, is used, then the disclaimer can be provided directly in the system.},
% for example: The authors have no competing interests to declare that are
% relevant to the content of this article. Or: Author A has received research
% grants from Company W. Author B has received a speaker honorarium from
% Company X and owns stock in Company Y. Author C is a member of committee Z.
\end{credits}
%
% ---- Bibliography ----
%
% BibTeX users should specify bibliography style 'splncs04'.
% References will then be sorted and formatted in the correct style.
%
\bibliographystyle{splncs04}
\bibliography{ijcar}
%
\end{document}
